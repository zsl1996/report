%%%%%%%%%%%%%%%%%%%%%%%%%%%%%%%%%%%%%%%%%%%%%%%%%%%%%%%%%%%%%%%%
%
%  Template for homework of Introduction to Machine Learning.
%
%  Fill in your name, lecture number, lecture date and body
%  of homework as indicated below.
%
%%%%%%%%%%%%%%%%%%%%%%%%%%%%%%%%%%%%%%%%%%%%%%%%%%%%%%%%%%%%%%%%


\documentclass[11pt,letter,notitlepage]{article}
%Mise en page
\usepackage[left=2cm, right=2cm, lines=45, top=0.8in, bottom=0.7in]{geometry}
\usepackage{fancyhdr}
\usepackage{fancybox}
\usepackage{graphicx}
\usepackage{pdfpages} 
\renewcommand{\headrulewidth}{1.5pt}
\renewcommand{\footrulewidth}{1.5pt}
\pagestyle{fancy}
\newcommand\Loadedframemethod{TikZ}
\usepackage[framemethod=\Loadedframemethod]{mdframed}

\usepackage{amssymb,amsmath}
\usepackage{amsthm}
\usepackage{thmtools}

\setlength{\topmargin}{0pt}
\setlength{\textheight}{9in}
\setlength{\headheight}{0pt}

\setlength{\oddsidemargin}{0.25in}
\setlength{\textwidth}{6in}

%%%%%%%%%%%%%%%%%%%%%%%%
%% Define the Exercise environment %%
%%%%%%%%%%%%%%%%%%%%%%%%
\mdtheorem[
topline=false,
rightline=false,
leftline=false,
bottomline=false,
leftmargin=-10,
rightmargin=-10
]{exercise}{\textbf{Exercise}}
%%%%%%%%%%%%%%%%%%%%%%%
%% End of the Exercise environment %%
%%%%%%%%%%%%%%%%%%%%%%%

%%%%%%%%%%%%%%%%%%%%%%%
%% Define the Solution Environment %%
%%%%%%%%%%%%%%%%%%%%%%%
\declaretheoremstyle
[
spaceabove=0pt, 
spacebelow=0pt, 
headfont=\normalfont\bfseries,
notefont=\mdseries, 
notebraces={(}{)}, 
headpunct={:\quad}, 
headindent={},
postheadspace={ }, 
postheadspace=4pt, 
bodyfont=\normalfont, 
qed=$\blacksquare$,
preheadhook={\begin{mdframed}[style=myframedstyle]},
	postfoothook=\end{mdframed},
]{mystyle}

\declaretheorem[style=mystyle,title=Solution,numbered=no]{solution}
\mdfdefinestyle{myframedstyle}{%
	topline=false,
	rightline=false,
	leftline=false,
	bottomline=false,
	skipabove=-6ex,
	leftmargin=-10,
	rightmargin=-10}
%%%%%%%%%%%%%%%%%%%%%%%
%% End of the Solution environment %%
%%%%%%%%%%%%%%%%%%%%%%%

%% Homework info.
\newcommand{\posted}{\text{Sep. 10, 2018}}       			%%% FILL IN POST DATE HERE
\newcommand{\due}{\text{Sep. 17, 2018}} 			%%% FILL IN Due DATE HERE
\newcommand{\hwno}{\text{1}} 		           			%%% FILL IN LECTURE NUMBER HERE


%%%%%%%%%%%%%%%%%%%%
%% Put your information here %%
%%%%%%%%%%%%%%%%%%%
\newcommand{\name}{\text{Your Name}}  	          			%%% FILL IN YOUR NAME HERE
\newcommand{\id}{\text{PB00000000}}		       			%%% FILL IN YOUR ID HERE
%%%%%%%%%%%%%%%%%%%%
%% End of the student's info %%
%%%%%%%%%%%%%%%%%%%


\lhead{
	\textbf{\name}
}
\rhead{
	\textbf{\id}
}
\chead{\textbf{
		Homework: \hwno
}}


\begin{document}
\vspace*{-4\baselineskip}
\thispagestyle{empty}


\begin{center}
{\bf\large Introduction to Machine Learning}\\
{Fall 2018}
\end{center}

\noindent
Lecturer: Jie Wang  			 %%% FILL IN LECTURER HERE
\hfill
Homework \hwno             			
\\
Posted: \posted
\hfill
Due: \due
\\
Name: \name             			
\hfill
ID: \id						
\hfill

\noindent
\rule{\textwidth}{2pt}

\medskip

%%%%%%%%%%%%%%%%%%%%%%%%%%%%%%%%%%%%%%%%%%%%%%%%%%%%%%%%%%%%%%%%
%% BODY OF HOMEWORK GOES HERE
%%%%%%%%%%%%%%%%%%%%%%%%%%%%%%%%%%%%%%%%%%%%%%%%%%%%%%%%%%%%%%%%

\textbf{Notice, }to get the full credits, please present your solutions step by step.

\begin{exercise}[\textnormal{2pts}]
	Show that $(1-\epsilon)^m\leq e^{-m\epsilon}$, where $m\in \mathbb{N}$.
\end{exercise}

\begin{solution}
	Put your solutions here. 
\end{solution}


\begin{exercise}[Markov inequality \textnormal{2pts}]
	Let $X$ be a nonnegative randome variable on $\mathbb{R}$. Then, for all $t>0$, show that
	$$\mathbf{P}(X\geq t)\leq \frac{\mathbf{E}[X]}{t}.$$
\end{exercise}

\begin{solution}
	Put your solutions here. 
\end{solution}


\begin{exercise}[VC-dimension \textnormal{2pts}]
	Assume that the instance space $X=\mathbb{R}^2$ and the hypothesis space $H$ be the set of all linear threshold functions defined on $\mathbb{R}^2$. Find $VC(H)$ and prove it.
\end{exercise}

\begin{solution}
	Put your solutions here. 
\end{solution}

\begin{exercise}[Learning intervals \textnormal{4pts}] 
	Let the target concept class be $C=\{[a,b]:a<b, a,b\in\mathbb{R}\}$ and the hypotheses class $H=C$, and the version space be $VS_{H,D}$. Each $c\in C$ labels the points inside the interval positive and the others negative. A consistent learner will pick a consistent hypothesis---if any---$h\in H$ according to a set of i.i.d. samples $\{(x_1,c(x_1)),(x_2,c(x_2),\ldots,(x_m,c(x_m)\}$ that obey an unknown distribution $\mathcal{D}$. Please find 
	$$\mathbf{P}[\exists\, h\in VS_{H,D} \mbox{ and } error_{\mathcal{D}}(h)>\epsilon],$$
	and the corresponding sample complexity.
\end{exercise}


\begin{solution}
	Put your solutions here. 
\end{solution}
%%%%%%%%%%%%%%%%%%%%%%%%%%%%%%%%%%%%%%%%%%%%%%%%%%%%%%%%%%%%%%%%

\end{document}
