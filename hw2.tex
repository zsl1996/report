%%%%%%%%%%%%%%%%%%%%%%%%%%%%%%%%%%%%%%%%%%%%%%%%%%%%%%%%%%%%%%%%
%
%  Template for homework of Introduction to Machine Learning.
%
%  Fill in your name, lecture number, lecture date and body
%  of homework as indicated below.
%
%%%%%%%%%%%%%%%%%%%%%%%%%%%%%%%%%%%%%%%%%%%%%%%%%%%%%%%%%%%%%%%%


\documentclass[11pt,letter,notitlepage]{article}
%Mise en page
\usepackage[left=2cm, right=2cm, lines=45, top=0.8in, bottom=0.7in]{geometry}
\usepackage{fancyhdr}
\usepackage{fancybox}
\usepackage{graphicx}
\usepackage{pdfpages} 
\renewcommand{\headrulewidth}{1.5pt}
\renewcommand{\footrulewidth}{1.5pt}
\pagestyle{fancy}
\newcommand\Loadedframemethod{TikZ}
\usepackage[framemethod=\Loadedframemethod]{mdframed}

\usepackage{amssymb,amsmath}
\usepackage{amsthm}
\usepackage{thmtools}

%%%%%%%%%%%%%%%%%%%%%%%%
%%%%%% Define math operator %%%%%
%%%%%%%%%%%%%%%%%%%%%%%%
\DeclareMathOperator*{\argmin}{\bf argmin}
%%%%%%%%%%%%%%%%%%%%%%%


\setlength{\topmargin}{0pt}
\setlength{\textheight}{9in}
\setlength{\headheight}{0pt}

\setlength{\oddsidemargin}{0.25in}
\setlength{\textwidth}{6in}

%%%%%%%%%%%%%%%%%%%%%%%%
%% Define the Exercise environment %%
%%%%%%%%%%%%%%%%%%%%%%%%
\mdtheorem[
topline=false,
rightline=false,
leftline=false,
bottomline=false,
leftmargin=-10,
rightmargin=-10
]{exercise}{\textbf{Exercise}}
%%%%%%%%%%%%%%%%%%%%%%%
%% End of the Exercise environment %%
%%%%%%%%%%%%%%%%%%%%%%%

%%%%%%%%%%%%%%%%%%%%%%%
%% Define the Solution Environment %%
%%%%%%%%%%%%%%%%%%%%%%%
\declaretheoremstyle
[
spaceabove=0pt, 
spacebelow=0pt, 
headfont=\normalfont\bfseries,
notefont=\mdseries, 
notebraces={(}{)}, 
headpunct={:\quad}, 
headindent={},
postheadspace={ }, 
postheadspace=4pt, 
bodyfont=\normalfont, 
qed=$\blacksquare$,
preheadhook={\begin{mdframed}[style=myframedstyle]},
	postfoothook=\end{mdframed},
]{mystyle}

\declaretheorem[style=mystyle,title=Solution,numbered=no]{solution}
\mdfdefinestyle{myframedstyle}{%
	topline=false,
	rightline=false,
	leftline=false,
	bottomline=false,
	skipabove=-6ex,
	leftmargin=-10,
	rightmargin=-10}
%%%%%%%%%%%%%%%%%%%%%%%
%% End of the Solution environment %%
%%%%%%%%%%%%%%%%%%%%%%%

%% Homework info.
\newcommand{\posted}{\text{Oct. 8, 2018}}       			%%% FILL IN POST DATE HERE
\newcommand{\due}{\text{Oct. 15, 2018}} 			%%% FILL IN Due DATE HERE
\newcommand{\hwno}{\text{2}} 		           			%%% FILL IN LECTURE NUMBER HERE


%%%%%%%%%%%%%%%%%%%%
%% Put your information here %%
%%%%%%%%%%%%%%%%%%%
\newcommand{\name}{\text{Shilong Zhang}}  	          			%%% FILL IN YOUR NAME HERE
\newcommand{\id}{\text{PB14214061}}		       			%%% FILL IN YOUR ID HERE
%%%%%%%%%%%%%%%%%%%%
%% End of the student's info %%
%%%%%%%%%%%%%%%%%%%


\lhead{
	\textbf{\name}
}
\rhead{
	\textbf{\id}
}
\chead{\textbf{
		Homework \hwno
}}


\begin{document}
\vspace*{-4\baselineskip}
\thispagestyle{empty}


\begin{center}
{\bf\large Introduction to Machine Learning}\\
{Fall 2018}\\
University of Science and Technology of China
\end{center}

\noindent
Lecturer: Jie Wang  			 %%% FILL IN LECTURER HERE
\hfill
Homework \hwno             			
\\
Posted: \posted
\hfill
Due: \due
\\
Name: \name             			
\hfill
ID: \id						 x
\hfill

\noindent
\rule{\textwidth}{2pt}

\medskip





%%%%%%%%%%%%%%%%%%%%%%%%%%%%%%%%%%%%%%%%%%%%%%%%%%%%%%%%%%%%%%%%
%% BODY OF HOMEWORK GOES HERE
%%%%%%%%%%%%%%%%%%%%%%%%%%%%%%%%%%%%%%%%%%%%%%%%%%%%%%%%%%%%%%%%

\textbf{Notice, }to get the full credits, please present your solutions step by step.

\begin{exercise}[Projection Operator \textnormal{4pts}]
	For a nonempty, closed, and convex set $S\subseteq\mathbb{R}^n$, the projection of an arbitrary point $x\in\mathbb{R}^n$ onto $S$ is defined by
	\begin{align}
		P_S(x)=\argmin_{z\in S}\,\|x-z\|_2.
	\end{align}
	Show that 
	\begin{enumerate}
		\item (1pt) $P_S(x)$ always exists and is unique;
		\item (2pt) $y=P_S(x)$ if and only if $y\in S$ and 
		\begin{align*}
			\langle z-y, x-y\rangle\leq0,\,\forall\,z\in S.
		\end{align*}
		\item (1pt) for all $x,y\in\mathbb{R}^n$,
		\begin{align*}
			\|P_S(x)-P_S(y)\|_2\leq \|x-y\|_2.
		\end{align*}
	\end{enumerate}
	
\end{exercise}

\begin{solution}
	
	
	(1) 1. if $s \in S $,the conclusion is obviously ok. \\
\indent 2. if $s \notin S $ .we consider a set that $B = \{z \in R^n|  ||z||<1 \}$,we can find a $\beta$ big enough to make $D = S \bigcap (x + \beta B )\neq \emptyset $ ,and D is a bounded closed set,and $||x-z||_2$ is a continuous function,so that the function must have a minima at $D \in S$,so we prove $P_s(x)$ always exists.
\indent	To prove it is unique,wo assume $x_1$ can make  $||x-z||_2$ get the minima which equal r, if there is another $x_2$ can make can make  $||x-z||_2$get the minima too,so the $x_3= \frac{x_1+x_2}{2} $,and $x_3$ must belong to S for S is a convex set.
and $x_1$ , $x_2$, $z$ can construct a isosceles triangle,and we must have $||x_3 -z|| < r$,which is contradictory with r is the minima.
So the it is unique. \\
\indent	(2)   we first prove if  $y=P_S(x)$,there must have $<z -y,x -y> \leq 0,\,\forall\,z\in S.$ ,we proof by contradiction,if there is a  $z\in S$,can make  $<z -y,x -y> > 0$,we can find a $z_1 = y + t(z-y)$,and t is a constant,we now prove $ ||z_1 - x||_2 < ||y -x||_2$when t is small enough.
	$$ ||z_1 - x||^2 = ||z_1 -y||^2 + ||y -x||^2 + 2<z_1 -y, y-x> = t^2||z-y||^2 + || y-x||^2   - 2t <z-y,y-x> $$,when t is small enough ,we can get $$ ||z_1 - x||^2 <  ||y -x||^2 $$ which is contradiction with  $y=P_S(x)$.
	then we prove if $<z -y,x -y> \leq 0,\,\forall\,z\in S.$,we have  $y=P_S(x)$.we can easy prove it by the proof above,for any $z_{n-1}\in S$, we always can find a ponit $z_n = y + t(z_{n-1}-y)$ can make $||z_n -y|| $smaller than $||z_{n-1} -y ||$,and the limit of $Z_n$ is $y$ ,so we prove it. \\
\indent	(3)  
\end{solution}


\begin{exercise}[Separation Theorems \textnormal{6pts}]
	Let $S_1, S_2\subseteq\mathbb{R}^n$ be nonempty convex sets.
	Show that
	\begin{enumerate}
		\item (2pt) if $S_1$ is closed, then for $x\notin S_1$, there exists a nonzero vector $v\in\mathbb{R}^n$ and $\epsilon>0$ such that
		\begin{align*}
		\langle v,y\rangle\leq\langle v, x\rangle-\epsilon,\,\forall\,y\in S_1;
		\end{align*}
		
		\item (1pt) if $S_1$ and $S_2$ are closed, $S_1$ is bounded and $S_1\bigcap S_2=\emptyset$, then there exists a nonzero vector $v\in\mathbb{R}^n$ and $\epsilon>0$ such that
		\begin{align*}
			\langle v,x_1\rangle\leq\langle v,x_2\rangle-\epsilon,\,\forall\,x_1\in S_1, x_2\in S_2;
		\end{align*}
		
		\item (2pt) for $x\notin S_1$, there exists a nonzero vector $v\in\mathbb{R}^n$ such that
		\begin{align*}
		\langle v,y\rangle\leq\langle v, x\rangle,\,\forall\,y\in S_1;
		\end{align*}
		
		\item (1pt) if $S_1\bigcap S_2=\emptyset$, then there exists a nonzero vector $v\in\mathbb{R}^n$ such that
		\begin{align*}
		\langle v,x_1\rangle\leq\langle v,x_2\rangle,\,\forall\,x_1\in S_1, x_2\in S_2.
		\end{align*}
	\end{enumerate}
	
\end{exercise}

\begin{solution}
	1.we assume that $x_1 = P_s(x)$,and make $ v = x - x_1$,we have prove the $	\langle v,y-x_1	\rangle \leq 0$ ,so $\langle v,y \rangle> < \langle v,x_1 \rangle $ ,and 
$\langle v,x-x_1\rangle > 0$,so $\langle v,x \rangle > \langle v,x_1 \rangle $, so$\langle v,y \rangle \leq \langle v,x \rangle -\epsilon ,\,\forall y \in S_1$. \\
\indent	2. we  construct a new set $$ S_0 = S_1 -S_2 = \{x_1 - x_2|x_1 \in S_1 ,\, S_2 \in S_2 \}$$
	and we know S is convex,closed and $S \neq \emptyset$,and $S_1 \bigcap S_2 = \emptyset$ ,so $ O \notin S$,so there must exist v can make $$ \langle v,x-0 \rangle > 0,\,\forall x \in S \text{the conclusion of question 1} $$ \\ \indent we prove it.
	\indent \\
	\indent3.when $S_1$ is closed ,we have prove it,and when $S_1 $is't closed,what we can define $P_s(x)$ as a limit point of  $z_k \in S_1$ which can make $ ||z_k - x||  < ||z_{k-1} - x||,$,and we easy to know $P_s(x) \in S_1^c$because $S_1 $is't closed,so the proof of 1 the x can be $x_1$,so $ \langle v,x - x_1 \rangle \> 0$ should change to$ \langle v,x - x_1 \rangle \geq 0 $,the = will be ok when x in the border of set $S_1$. \\
 \indent	so we get $\langle v,y\rangle\leq\langle v, x\rangle,\,\forall\,y\in S_1$
\\
\indent 4. we can use the conclusion of 3 and do the same thing as 2 and we can prove it easily.

\end{solution}



%%%%%%%%%%%%%%%%%%%%%%%%%%%%%%%%%%%%%%%%%%%%%%%%%%%%%%%%%%%%%%%%

\end{document}
